\documentclass[compress,aspectratio=43]{beamer}

\usepackage[english]{babel}
\usepackage[english]{isodate}
\isodate

\usepackage{minted}

\usetheme{vertex}

\linespread{1.1}

\title{C++11}
\subtitle{Writing modern scientific software}
\date{\today}
\author{Igor Babuschkin\\ Jens Winkelmann}

\begin{document}

\maketitle

\begin{frame}{Complementary exercises}
  Can be found at \\
  \begin{center}
  \url{github.com/ibab/cpp11-talk}
  \end{center}
  in the \texttt{exercises} directory.
\end{frame}

\begin{frame}{Why C++11?}
  Can make your code
  \begin{itemize}
    \item Simpler
    \item Shorter
    \item Safer
    \item Faster
  \end{itemize}
\end{frame}

\begin{frame}[fragile]{auto}
  Let the compiler infer the type of a variable
  \begin{minted}[linenos]{cpp}
std::vector<std::vector<double>>::iterator it
  = data.begin();
// VS
auto it = data.begin();
  \end{minted}
  Useful to keep your code readable when templates are involved.
\end{frame}

\begin{frame}[fragile]{auto (cont.)}
  Doesn't always make sense:
  \begin{minted}[linenos]{cpp}
double x = 1;
auto   y = 1;
  \end{minted}
  Question: What is the value of this expression?
  \begin{minted}[linenos]{cpp}
y / 2;
  \end{minted}
\end{frame}

\begin{frame}[fragile]{Lambda expressions}
  Quickly define functions as if they were values and pass them around:
  \begin{minted}[linenos]{cpp}
func = [](double x) { return 42 * x; };
dosomething(func);
  \end{minted}
\end{frame}

\begin{frame}[fragile]{Lambda expressions (cont.)}
  Very useful: Injecting variables into the scope of your function.
  \begin{minted}[linenos]{cpp}
for (int q = 1; q < 10; q++) {
  newfunc = [q](double x) {
    return pow(x, q);
  }
  dosomething(newfunc);
}
  \end{minted}
\end{frame}

\begin{frame}[fragile]{Lambda expressions (cont.)}
  Lambda functions have the type \texttt{std::function} (\texttt{\#include<functional>})\\
  You will need this when taking/returning a lambda function from a regular function.
  \begin{minted}[linenos]{cpp}
void dosomething(std::function<double(double)>);
  \end{minted}
\end{frame} 

\begin{frame}{std containers}
  More emphasis is placed on std containers
  \begin{itemize}
    \item vector: extensible array
    \item set: unordered container with fast lookup
    \item map: extensible key/value store with fast lookup
    \item tuple: fixed size collection of different objects
  \end{itemize}
  Should be preferred to simple data types (like arrays) for their convenience and safety with little to no performance loss.
\end{frame}

\begin{frame}{Smart pointers}
  Pointers are the source of many bugs, but can't be avoided completely.
  C++11 introduces wrappers that help managing them when they are really necessary:
  \begin{itemize}
    \item Interfacing with pointer-based code
    \item Polymorphism
  \end{itemize}
\end{frame}

\begin{frame}[fragile]{Smart pointers (cont.)}
  \begin{minted}{cpp}
std::unique_ptr<TestObject> obj(new TestObject());
  \end{minted}
  \begin{itemize}
    \item Use \texttt{*obj} and \text{obj->bla()} as usual.
    \item A function takes a pointer and takes over ownership? $\rightarrow$ use \texttt{obj.release()}
    \item It does not take ownership? $\rightarrow$ \texttt{obj.get()}
  \end{itemize}
  If you copy one \texttt{unique\_ptr} into another, the original one loses his reference. \\
  The underlying object is deleted if \texttt{unique\_ptr} goes out of scope. \\
  $\Rightarrow$ No memory leaks if used correctly!
\end{frame}

\begin{frame}{Smart pointers (cont.)}
  Also new: \texttt{shared\_ptr}. \\
  Solves the problem of how to manage many pointers pointing to a single object.

  A reference count is maintained and the object is deleted when it reaches 0.
\end{frame}

\begin{frame}[fragile]{foreach loops}
  Every object that implements \texttt{begin()} and \texttt{end()} can be iterated over with foreach syntax:
  \begin{minted}[linenos]{cpp}
std::vector<double> data = {1.0, 1.2, 1.4};
for(auto &x: data) {
  std::cout << x << std::endl;
}
  \end{minted}
  Compare this to
  \begin{minted}[linenos]{cpp}
for(std::vector<double>::iterator it=data.begin();
      it!=data.end(); it++) {
  std::cout << *it << std::endl;
}
  \end{minted}
\end{frame}

\begin{frame}[fragile]{Random number generators}
  Very useful interface for sampling from discrete or continuous distributions.\\
  Prepare random generator
  \begin{minted}{cpp}
std::random_device rd;
std::mt19937 rand(rd());
  \end{minted}
  Prepare distribution
  \begin{minted}{cpp}
std::normal_distribution dist(0.0, 1.0);
  \end{minted}
  Generate values
  \begin{minted}{cpp}
dist(rand);
  \end{minted}
\end{frame}


\end{document}

